% !TEX program = xelatex

\documentclass{resume}
%\usepackage{zh_CN-Adobefonts_external} % Simplified Chinese Support using external fonts (./fonts/zh_CN-Adobe/)
%\usepackage{zh_CN-Adobefonts_internal} % Simplified Chinese Support using system fonts

\begin{document}
\pagenumbering{gobble} % suppress displaying page number

\name{Jiachen Lu}

\basicInfo{
  \email{jiachen\_lu1999@163.com} \textperiodcentered\ 
  \phone{(+49) 0157 3720 8089} \textperiodcentered\ 
  \linkedin[Jiachen Lu]{https://www.linkedin.com/in/jiachen-lu-08b85a1bb/} \textperiodcentered\ 
  \github[Jiachen Lu]{https://github.com/LuckyMax0722}
}

\section{\faGraduationCap\ Education}
\edusubsection
  {\textbf{Technical University of Munich}, Munich, Germany}{10/2022 -- 06/2025}
\edusubsection
  {\textit{Master of Science (M.Sc.)}\ in \textbf{Robotics, Cognition, Intelligence (RCI)}}{GPA: 1.7/1.0 (Lower is better)}
\coursesubsection
  {Core Curriculum:}{Artificial Intelligence, Robotics, Machine Learning, Machine Learning (graph and sequence data),}
\coursesubsection
  {}{Deep Learning, Computer Vision (multi-view geometry/detection, segmentation and tracking),}
\coursesubsection
  {}{Advanced Driver Assistance System (ADAS), Autonomous Driving (AD) Software Development}
\vspace{0.2cm}

\edusubsection
  {\textbf{Coburg University of Applied Sciences}, Coburg, Germany}{10/2020 -- 04/2022}
\edusubsection
  {\textit{Bachelor of Engineering (B.Eng.)}\ in \textbf{Automobile Engineering (AE)}}{GPA: 2.0/1.0 (Lower is better)}
\coursesubsection
  {Core Curriculum:}{Vehicle dynamics, Mechatronics}
\vspace{0.2cm}

\edusubsection
  {\textbf{Tongji University}, Shanghai, China}{10/2017 -- 04/2022}
\edusubsection
  {\textit{Bachelor of Engineering (B.Eng.)}\ in \textbf{Vehicle Engineering \& After-Sales Services}}{GPA: 2.6/1.0 (Lower is better)}
\coursesubsection
  {Core Curriculum:}{Advanced Mathematics, Physics, Mechanics, Electricity, Control Theory, Sensors and Actuators}
\vspace{0.2cm}


\section{\faUsers\ Working Experience}
\worksubsection
  {\textbf{Porsche Engineering Services GmbH}, Moensheim, Germany}{04/2024 -- 09/2024}
\workpossubsection
  {\textbf{ADAS Test and Development Engineer}}{Internship}

\ \textit{Tech Stack:\ Python/PyTorch/PyTorch Lightning/Jira/Confluence/Codebeamer}
\begin{itemize}
  \item Support the AI team in their daily work and complete sub-projects related to Deep Learning. Implemented an inference and visualization script for \textbf{LATR-based 3D Lane Line Detection} to detect and classify road boundaries and fences. Reproducing \textbf{SignParser-based Traffic Sign Understanding} for recognizing and understanding text-based traffic signs. Reproducing \textbf{Metric3D-based Traffic Sign Reconstruction} for reconstructing 3D traffic signs. Based on the above outputs generate real-world 3D modeling in real time in the simulator
  \item Responsible for functional/unit/integration testing of \textbf{Mobileye SuperVision's L2++ Advanced driver-assistance systems (ADAS)}. Constructed test reference routes based on Google Map and actual driving experience, and designed test cases based on multi-dimensional test evaluation matrix and actual weather and road conditions. Provide on-site support and accompanying tests for highway and urban road tests, and record test conditions in real time.
  \item Provided test system maintenance and development for Macan 4 with IAV and Bosch Parking Assist, on-site support and accompaniment for functional tests such as ePark/TPA/RA, and real-time documentation of test status and test case passes. And develop the video detection model for detecting the status of HMI buttons based on \textbf{YOLOv9}, which is used to compare with the button signals on the CAN-BUS.
  \item Support the ADAS Driving and Parking team in their daily development and testing activities
\end{itemize}
\vspace{0.15cm}

\worksubsection
  {\textbf{Daimler Trucks AG}, Stuttgart, Germany}{10/2021 -- 03/2022}
\workpossubsection
  {\textbf{Charging System Test and Development Engineer}}{Bachelor's Degree Thesis}

\ \textit{Thesis title:\ Development of a restbussimulation for charging system control units and software modules as well as test concepts}

\ \textit{Tech Stack:\ CAPL/Vector CANoe/Hardware-in-the-loop simulation (HiLs)/Restbussimulation}
\begin{itemize}
  \item Design and development of V-models, \textbf{Hardware-in-the-Loop simulation (HiLs)} and \textbf{Restbussimulation}, for the testing of the charging system components of the eActros electric trucks
  \item Write, Extend and Optimize existing test cases based on existing test frameworks and ECU development standards
  \item Introduce the concept of \textbf{Key Performance Indicators (KPIs)}, develop evaluation criteria and tools for test case automation, and evaluate existing test cases
  \item Write implementation scripts for automation test cases based on \textbf{CAPL} and \textbf{CANoe}, design and build corresponding script configuration and visual user interface
\end{itemize}
\vspace{0.15cm}

\worksubsection
  {\textbf{Daimler Trucks AG}, Esslingen am Neckar, Germany}{05/2021 -- 10/2021}
\workpossubsection
  {\textbf{High Voltage (HV) Component Test and Development Engineer}}{Internship}

\ \textit{Tech Stack:\ CAPL/Vector CANape/Vector CANalyzer}
\begin{itemize}
  \item Supporting teams in the daily development and testing of HV resistor assemblies in the powertrain of eActros electric trucks
  \item Design of test concepts and coordination of test plans for the eActros summer road function tests. Provide \textbf{on-site support} and accompany the tests during the testing
  \item Design and build a \textbf{GUI} based on \textbf{CAPL} and \textbf{CANape} for online monitoring of the operational status of specific components in the test vehicle. Write component test scripts to monitor, collect and analyze test data online by monitoring the CAN bus
  \item Develop and write automated data mining scripts for offline evaluation of specific components of test vehicles based on CAPL and CANape's \textbf{data mining} capabilities
\end{itemize}
\vspace{0.15cm}

\section{\faFolder\ Project experience}
\worksubsection{\textbf{Master Thesis: 3D Semantic Scene Completion}}{08/2024 -- 06/2025}
\ \textit{Tech Stack:\ Python/Pytorch \hfill Project Link:\ }
\begin{itemize}
  \item \textbf{MonoScene-based} and \textbf{VoxFormer-based}
\end{itemize}

\worksubsection{\textbf{TOD2D: Traffic Object target Detection and classification for 2D images}}{03/2024 -- 07/2024}
\ \textit{Tech Stack:\ Python/Pytorch/OpenCV/YOLOv5-v9/DETR/SwinT/ResNet/EfficientNet \hfill Project Link:\ \href{https://github.com/LuckyMax0722/TOD2D}{TOD2D}}
\begin{itemize}
  \item Data cleansing, data augmentation and creation of dataset in \textbf{YOLO/COCO} format based on \textbf{nuImages} 2D image dataset
  \item Target detection of images in nuImages dataset using \textbf{YOLOv5-v9} belonging to One-Stage and Transformer-based \textbf{DETR/SwinT}
  \item Using \textbf{OpenCV} and pre-trained \textbf{YOLOv9} to extract and preclassify the target objects for the traffic light dataset \textbf{DTLD/BSTLD} and the traffic sign dataset \textbf{GTSRB/TT100K}, resize the images of the target objects and create the YOLO-format dataset
  \item Using the manually created traffic light and traffic sign datasets, pre-training the classification headers for classifying  the type and color of traffic lights as well as the content of traffic signs based on \textbf{ResNet50} and \textbf{EfficientNet b3} were used as Second-Stage classifiers for \textbf{YOLOv9}
  \item Compared to direct training \textbf{YOLOv9}, \textbf{TOD} has improved training speed by \textbf{65\%}, reduced hardware requirements by \textbf{25\%} and improved ACC by about \textbf{12\%}
\end{itemize}

\worksubsection{\textbf{End-to-end learning for self-driving cars}}{10/2023 -- 03/2024}
\ \textit{Tech Stack:\ Python/Pytorch/Pytorch Lightening/OpenCV/ResNet/ViT/GRU \hfill Project Link:\ \href{https://github.com/LuckyMax0722/SelfDrivingCars}{SelfDrivingCars}}
\begin{itemize}
  \item Based on Unity's car driving simulator, manually sampling the training data and utilizing \textbf {OpenCV} to clean, filter, process and augment the raw image data
  \item Using \textbf{ResNet50} as an image feature learning backbone module to realize direct steering angle prediction using vehicle front images, i.e., end-to-end learning
  \item In the ablation experiments, the performance of different network architectures in realizing end-to-end learning are tested, including \textbf{ResNet50}, \textbf{ResNet50+GRU} and \textbf{ViT}
  \item Compared to other models, the training and inference speed of ResNet50 is improved by \textbf{35\%}, and the autonomous driving model trained based on ResNet50 realizes the high speed of a small car in the driving simulator with \textbf{0} collision
\end{itemize}

\worksubsection{\textbf{SoftCap: Generating Dense Descriptions for 3D Point Cloud using Sparse Convolution}}{04/2023.04 -- 09/2023}
\ \textit{Tech Stack:\ Python/Pytorch/Pytorch Lightening/C++/SoftGroup/GNN/GRU/Attention \hfill Project Link:\ \href{https://github.com/LuckyMax0722/SoftCap}{SoftCap}}
\begin{itemize}
  \item Applying \textbf{SoftGroup} as the detection backbone module in 3D point cloud scenarios to implement a soft grouping mechanism on point cloud data for instance proposal generation and classification
  \item Constructing \textbf{GNN} based on physical relationships between instances in the 3D point cloud scene, and obtaining as well as learning spatial features from instance to instance through the message passing algorithm
  \item Generating descriptions of instance features and their spatial attributes in the 3D point cloud scene based on augmented object features by means of a multilayer \textbf{GRU module} and \textbf{attention mechanism}
  \item In the process of training the model, supervised learning based on \textbf{Teacher Forcing} and reinforcement learning based on \textbf{Self-Critical} are used
  \item In the ScanRefer dataset, SoftCap performs well in localizing and describing objects in the 3D point cloud scene, \textbf{mAP@0.5IoU} reaching \textbf{57.38} and \textbf{CIDEr@0.5IoU} reaching \textbf{36.27}. Compared to previous work, SoftCap's performance improves \textbf{140\%}
\end{itemize}

\section{\faStar\ Honors and Awards}
\begin{itemize}
  \item \itemsubsection{\textbf{Phoenix Contact Scholarship}}{}{}{}{}{}{2020.09}
\end{itemize}


\section{\faCogs\ IT Skills}

\begin{itemize}
  \item \itsubsection{\textbf{Programming Languages:}}{Python, C++, CAPL, Matlab/Simulink}
  \item \itsubsection{\textbf{Commonly used tools:}}{Pytorch, Pytorch Lightening, NumPy, OpenCV, Pandas, Git, Docker}
  \item \itsubsection{\textbf{Commonly used Software:}}{Word/Excel/PowerPoint, Vector CANoe/CANape/CANalyzer, AutoCAD, CATIA V5}
\end{itemize}

\section{\faLanguage\ Language Skills}
% increase linespacing [parsep=0.5ex]
\begin{itemize}
  \item \itemsubsection{\textbf{English (C1):} IELTS}
  {\textbf{Overall: 7}}{Listening: 8}{Reading: 7}{Writing: 6.5}{Speaking: 6}{2022.01}

  \item \itemsubsection{\textbf{German (C1):} TestDaF}
  {\textbf{Overall: 15}}{Listening: 3}{Reading: 4}{Writing: 4}{Speaking: 4}{2021.12}
\end{itemize}

\section{\faWrench\ Other Skills}
% increase linespacing [parsep=0.5ex]
\begin{itemize}
  \item \itsubsection{\textbf{Driving License:}}{German B197 license, Chinese C1 license}
\end{itemize}
\end{document}
